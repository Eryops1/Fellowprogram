\documentclass[11pt,a4paper]{article}
\usepackage[utf8]{inputenc}
\usepackage[ngerman]{babel}
\usepackage[left=2.54cm,right=2.54cm,top=2.54cm,bottom=2.54cm]{geometry}
\usepackage{paralist}
\usepackage[colorlinks=true, urlcolor=blue]{hyperref}
%\renewcommand*\rmdefault{iwona}
%\renewcommand*\rmdefault{lmss}
\begin{document}

\title{Abschlussbericht Fellow-Programm Freies Wissen}
\author{Melanie Tietje}
\maketitle


\section{Infos zum Forschungsvorhaben}% max. 3000 Zeichen
\subsection{Zusammenfassung und Ergebnisse} %​ Beschreibe die abschließenden Ergebnisse deines Forschungsvorhabens anhand deiner individuellen Roadmap. Wurden die im Rahmen des Fellow-Programms formulierten Forschungsziele erreicht oder gab es Änderungen? Wenn ja, welche? ​ Bitte beschreiben.
Die Recherche zur Akzeptanz von Preprints bei diversen Journals, die für meinen Forschungszweig von Interesse sind, wurde bereits zu Beginn abgeschlossen und die Ergebnisse im Zwischenbericht erläutert. Zusätzlich sind mir noch zwei neue Open Access Journals bekannt geworden (prerint policy für Facets in wiki ergänzen - oder ist das canadian publishing?)

In meinem Projekt habe ich alle Bereiche abgedeckt, die ich mir in meiner Roadmap vorgenommen habe, allerdings unter leicht veränderten Umständen. So ist das Manuscript zu meinem Forschungsvorhaben noch nicht bei Plos o.ä. submitted, weil sich das Forschungsprojekt nicht ganz reibungslos gestaltet hat. Probleme sind dabei rein inhaltlicher Natur gewesen, nicht organisatorischer. Dies sehe ich jedoch als völlig normalen Bestandteil der naturwissenschaftlichen Forschung. Den Aspekt des Open Peer Review konnte ich trotzdem durch ein anderes Projekt abdecken, das ich bereits zu Beginn des Projekts im September in einer ersten Manuskriptfassung hatte und im Januar bei Royal Society Open Science eingereicht habe. Das Manuskript befindet sich seit dem 17. Januar im Open Peer Review Prozess (Stand 8. Februar). An dem Plan, mein derzeitiges Manuskript über einen Preprint Server einzureichen, hat sich jedoch nichts geändert und ich werde dabei genau so vorgehen wie geplant.

Die Reproduzierbarkeit und Öffnung für andere Wissenschaftler habe ich durch die Projektwebsite mit regelmäßigen Updates erreicht.

\subsection{Welchen Beitrag zum Themengebiet Open Science hat dein Forschungsprojekt geleistet?}% Bitte beschreiben.
\begin{compactitem}
\item Erfahrung mit Open Publishing (Open Access + Review) im Bereich Paläobiologie
\item Erfahrung mit Projektwebsites + Feedback
\item Förderung des Bewusstseins über Open Science unter den Nachwuchswissenschaftlern am Museum für Naturkunde
\end{compactitem}



\section{Zusammenarbeit mit Fellows und MentorInnen}% max. 3000 Zeichen
\subsection{Zusammenarbeit mit meinem Mentor}% Wie regelmäßig fand ein Austausch statt? Wie/ über welche Kanäle habt ihr kommuniziert? Wie hilfreich war der Austausch für dich/ dein Forschungsvorhaben? Was hättest du dir für die Zusammenarbeit noch gewünscht?
Wir kommunizieren regelmäßig über Email, Skype, oder auch im GitHub repo. Der Austausch fand etwas seltener als einmal wöchtenlich statt. Daniel hat mich bei jedem Gespräch mit immer neuen Informationen über Open Science Optionen versorgt. 


\subsection{Zusammenarbeit mit meinen Fellow-Partnern}%: Wie regelmäßig findet ein Austausch statt? Wie/ über welche Kanäle kommuniziert ihr? Wie hilfreich ist der Austausch für dich/ dein Forschungsvorhaben? Was wünschst du dir für die künftige Zusammenarbeit?


\subsection{Austausch mit anderen Fellows}%: Inwiefern findet auch Austausch mit den anderen Fellows statt? Was würdest du dir ggf. an Austausch wünschen?



\section{Kommunikation und Vernetzung}% max. 2000 Zeichen
\subsection{Kommunikationsaktivitäten}% Welche ​ Kommunikationsaktivitäten mit Bezug zum Fellowprogramm/ Open Science (Teilnahme an Veranstaltungen, Fachbeiträge, Blogposts) hast du initiiert?

\begin{compactitem}
\item Blogpost Wikimedia Deutschland
\item Community digest für englischssprachigen Wikimedia Blog auf Initiative von Daniel (Nachfrage beim englischen Wikimedia Blog)
\item Workshop Museum für Naturkunde Open Science (vermutlich 10. März 2017)
\end{compactitem}


\subsection{Kontakte Open-Science-Community}%Haben sich neue Kontakte oder Austauschmöglichkeiten mit Vertreterinnen und Vertretern aus der Open-Science-Community gebildet? 


\subsection{Kontakte Wikimedia-Communitys}%Haben sich neue Kontakte oder Austauschmöglichkeiten mit Vertreterinnen oder Vertretern mit den Wikimedia-Communitys gebildet?

\subsection{Vernetzungsmöglichkeiten} %Welche ​ Vernetzungsmöglichkeiten ​ (Formate etc.) kannst du dir nach Abschluss des Programms vorstellen, um mit den Fellows, Mentorinnen und Mentoren sowie Wikimedia Deutschland im Austausch zu bleiben? Bitte Beispiele benennen.
GitHub ist eine Option für alle mit Account, und auch noch über Projekte zu berichten, die evtl. erst nach dem offiziellen Ende des Programms abgeschlossen werden. Facebook ist eine bequeme Option, für die meisten vermutlich besser in den Alltag integriert, jedoch weniger inhaltlich orientiert.




\section{Förderung von Open Science} % max. 4000 Zeichen
\subsection{Neue Open-Science-Initiativen am Institut?}
%Sind (neue)​ Open Science-Initiativen​ im Rahmen des Fellow-Programms an deiner wissenschaftlichen Einrichtung entstanden? Wenn ja, welche? Wenn nein, warum nicht?


\subsection{Open-Science Initiativen anstoßen}
%Welche Möglichkeiten siehst du, um (eigene) Open Science-Initiativen ​ an deiner wissenschaftlichen Einrichtung anzustoßen? ​Welche Herausforderungen und Chancen bestehen dort, um das Thema Open Science sichtbar/noch sichtbarer zu machen?
Herausforderung: Open Science ist nicht nur Citizen Science.


\subsection{Interesse an Open-Science im Umfeld} % Fachbereich, Umfeld?
- Ja, Arbeitsgruppe etc.

\subsection{Open Science in deiner Forschung}
%Inwiefern war das Fellow-Programm nützlich für dich, um aktiv(er) ​ Open Science in deiner Forschung ​ anzuwenden? Welche persönlichen Erfolge konntest du erzielen oder auch nicht? Wie möchtest du daran langfristig anknüpfen? ​
- massiver Zugewinn an Wissen zu Kommunikation von Projekten (Struktur, technische Umsetzbarkeit, Lizenzfragen)
- Erfahrung mit Open Peer-Review
- externe Motivation, etwas Neues tatsächlich auszuprobieren anstatt nur darüber nachzudenken


\subsection{Ansprechperson Open Science}
% Im Rahmen der Programmevaluation ​ planen wir, uns nach Ablauf der Programmlaufzeit nach dem aktuellen Stand von Open Science an den Institutionen der Fellows zu erkundigen. Wer ist hierfür an deiner Institution die entsprechende Ansprechperson?


\end{document}
