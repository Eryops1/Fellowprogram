\documentclass[11pt,a4paper]{article}
\usepackage[utf8]{inputenc}
\usepackage[ngerman]{babel}
\usepackage[left=2.54cm,right=2.54cm,top=2.54cm,bottom=2.54cm]{geometry}
\usepackage{paralist}
\usepackage[colorlinks=true, urlcolor=blue]{hyperref}
%\renewcommand*\rmdefault{iwona}
%\renewcommand*\rmdefault{lmss}
\begin{document}

\title{Abschlussbericht Fellow-Programm Freies Wissen}
\author{Melanie Tietje}
\maketitle


\section{Infos zum Forschungsvorhaben}% max. 3000 Zeichen
\subsection{Zusammenfassung und Ergebnisse} %​ Beschreibe die abschließenden Ergebnisse deines Forschungsvorhabens anhand deiner individuellen Roadmap. Wurden die im Rahmen des Fellow-Programms formulierten Forschungsziele erreicht oder gab es Änderungen? Wenn ja, welche? ​ Bitte beschreiben.
Die Recherche zur Akzeptanz von Preprints bei diversen Journals, die für meinen Forschungszweig von Interesse sind, wurde bereits zu Beginn abgeschlossen und die Ergebnisse im Zwischenbericht erläutert. Zusätzlich sind mir noch zwei neue Open Access Journals bekannt geworden (prerint policy für Facets in wiki ergänzen - oder ist das canadian publishing?)

In meinem Projekt habe ich alle Bereiche abgedeckt, die ich mir in meiner Roadmap vorgenommen habe, allerdings unter leicht veränderten Umständen. So ist das Manuscript zu meinem Forschungsvorhaben noch nicht bei einer Open Access Zeitschrift eingereicht, weil sich das Forschungsprojekt nicht ganz reibungslos gestaltet hat. Probleme sind dabei rein inhaltlicher Natur gewesen, nicht organisatorischer. Dies sehe ich jedoch als völlig normalen Bestandteil der naturwissenschaftlichen Forschung. Den Aspekt des Open Peer Review konnte ich trotzdem durch ein anderes Projekt abdecken, das ich bereits zu Beginn des Projekts im September in einer ersten Manuskriptfassung hatte und im Januar bei Royal Society Open Science eingereicht habe. Das Manuskript befindet sich seit dem 17. Januar im Open Peer Review Prozess, am 17. Februar habe ich das Feedback der Reviewer erhalten. Die revisions werden in spätestens 3 Wochen wieder eingereicht. Die Reviews werden bei Veröffentlichung mit veröffentlicht. An dem Plan, mein derzeitiges Manuskript über einen Preprint Server einzureichen, hat sich jedoch nichts geändert und ich werde dabei genau so vorgehen wie geplant.

Die Reproduzierbarkeit und Öffnung für andere Wissenschaftler habe ich durch die Projektwebsite mit regelmäßigen Updates umgesetzt.

Im Rahmen des Workshops am Museum, der am 10. März stattfinden wird, habe ich intern am Museum eine Diskussion angeregt und das Fellow Programm dabei bekannt gemacht. Das Hauptaugenmerk wird in dem Workshop auf Reproduzierbarkeit und Anwendbarkeit von Open Science im Forschungsalltag liegen.


\subsection{Welchen Beitrag zum Themengebiet Open Science hat dein Forschungsprojekt geleistet?}% Bitte beschreiben.
Ich habe Erfahrungen mit einem Open Peer Review Verfahren und mit dem Erstellen von Projektwebsites in GitHub Pages gesammelt sowie das Thema Open Science bei uns im Museum bekannter gemacht. Neben dem Sammeln von Erfahrungen mit für mich neuen Techniken und der Diskussion mit Kolleginnen hat mir das Projekt auch gezeigt, was in einem normalen Forschungsalltag möglich ist und wo die Grenzen, meist zeitlich bedingt, liegen. Kommunikation der eigenen Arbeit nach außen ist nicht nur aus altruistischen Gründen unbedingt zu empfehlen, der Zeitaufwand ist dabei aber auf jeden Fall einzukalkulieren. 



\section{Zusammenarbeit mit Fellows und MentorInnen}% max. 3000 Zeichen
\subsection{Zusammenarbeit mit meinem Mentor}% Wie regelmäßig fand ein Austausch statt? Wie/ über welche Kanäle habt ihr kommuniziert? Wie hilfreich war der Austausch für dich/ dein Forschungsvorhaben? Was hättest du dir für die Zusammenarbeit noch gewünscht?
Wir haben regelmäßig (möglichst einmal wöchentlich) über Email, Skype, oder auch in einem gemeinsamen Google Docs Informationen ausgetauscht. Daniel hat mich bei jedem Gespräch mit immer neuen Informationen über Open Science Optionen versorgt.


\subsection{Zusammenarbeit mit meinen Fellow-Partnern}%: Wie regelmäßig findet ein Austausch statt? Wie/ über welche Kanäle kommuniziert ihr? Wie hilfreich ist der Austausch für dich/ dein Forschungsvorhaben? Was wünschst du dir für die künftige Zusammenarbeit?
Wir haben zu Beginn sporadisch über GitHub und Email kommuniziert.
Der Austausch ist in der zweiten Hälfte des Programms zum Erliegen gekommen. Den Grund hierfür sehe ich darin, das die grundlegenden Mechanismen in den Projekten bereits etabliert wurden und der Fokus danach wieder vermehrt auf der inhaltlichen Arbeit liegt, zu der das Feedback von Forschern aus dem Fachbereich wichtiger ist. Um dem Kontakt aufrecht zu halten wäre es evtl. sinnvoll, einen Jour fixe zu einem festen Zeitpunkt (einmal monatlich) zu etablieren. Da unsere Forschungsgebiete aber doch recht unterschiedlich sind und jeder damit andere Ansätze für Open Science hat, halte ich einen ständigen Austausch nicht unbedingt für erforderlich.


\subsection{Austausch mit anderen Fellows}%: Inwiefern findet auch Austausch mit den anderen Fellows statt? Was würdest du dir ggf. an Austausch wünschen?
Austausch mit anderen Fellows fand nicht statt, allerdings fühlte ich mich über die Email-Liste bzw. Blog-Beiträge ausreichend über deren Vorhaben informiert.



\section{Kommunikation und Vernetzung}% max. 2000 Zeichen
\subsection{Kommunikationsaktivitäten}% Welche ​ Kommunikationsaktivitäten mit Bezug zum Fellowprogramm/ Open Science (Teilnahme an Veranstaltungen, Fachbeiträge, Blogposts) hast du initiiert?
Ich habe einen Blogbeitrag über mein Projekt für den deutschen Wikimedia Blog geschrieben sowie einen kurzen Community Digest Beitrag für den englisch-sprachigen Blog der Wikimedia. Ich  teile regelmäßig Informationen über Open Science Entwicklungen über meinen Facebook account. Den größten Einfluss düfte der Workshop am Museum haben, mit dem ich eine Diskussion über das Thema starten möchte. 


\subsection{Kontakte Open-Science-Community}%Haben sich neue Kontakte oder Austauschmöglichkeiten mit Vertreterinnen und Vertretern aus der Open-Science-Community gebildet? 
Ich stehe öfter in Kontakt mit einem englischen Kollegen, der sich seit Fertigstellung der Dissertation intensiv dem Thema Open Science widmet und als Communications Director bei ScienceOpen tätig ist.


\subsection{Kontakte Wikimedia-Communitys}%Haben sich neue Kontakte oder Austauschmöglichkeiten mit Vertreterinnen oder Vertretern mit den Wikimedia-Communitys gebildet?
Der Kontakt mit der Wikimedia Community gestaltet sich bei mir, abgesehen von Daniel, weiterhin etwas schwierig. Ich produziere weder Fotos noch andere Grafiken in meiner Arbeit, die für Commons interessant wären, sondern bin auf Modellierung und Statistik festgelegt. Ich halte die Plos Topic Pages jedoch für einen sehr guten Ansatz und möchte den Wikipedia Artikel Extinction risk auf diese Weise gestalten. 

\subsection{Vernetzungsmöglichkeiten} %Welche ​ Vernetzungsmöglichkeiten ​ (Formate etc.) kannst du dir nach Abschluss des Programms vorstellen, um mit den Fellows, Mentorinnen und Mentoren sowie Wikimedia Deutschland im Austausch zu bleiben? Bitte Beispiele benennen.
GitHub wäre eine Option für alle mit Account, um auch noch über Projekte zu berichten, die erst nach dem offiziellen Ende des Programms abgeschlossen werden. Dies wird für mein aktuelles Manuskript der Fall sein. Allerdings dürfte die Motivation, sich extra hierfür einen Account anzulegen gerade für Geisteswissenschaftler gering sein. Daher denke ich der Email-Verteiler sollte einfach bestehen bleiben. Facebook ist eine bequeme Option und für die meisten vermutlich besser in den Alltag integriert, jedoch weniger inhaltlich orientiert. 





\section{Förderung von Open Science} % max. 4000 Zeichen
\subsection{Neue Open-Science-Initiativen am Institut?}
%Sind (neue)​ Open Science-Initiativen​ im Rahmen des Fellow-Programms an deiner wissenschaftlichen Einrichtung entstanden? Wenn ja, welche? Wenn nein, warum nicht?
Neue Initiativen sind in dem Zeitraum des Programms nicht entstanden. Die Leibniz Gemeinschaft als Schirminstitution organisiert aber seit Beginn 2016 einen Publikationsfonds, um Open Access zu fördern.
In Gesprächen mit der Leitung des Fachbereichs Digitale Welt und Informationswissenschaft wurde mitgeteilt, dass es am Museum vor einigen Jahren bereits Bemühungen zu einer Open Science Richtlinie / Programm gab, allerdings konnten diese aus Kosten- und Personalgründen nicht weiter verfolgt werden. Das Kernproblem, wir auch in anderen Bereichen, ist das knappe Haushaltsbudget. Es wird in näherer Zukunft jedoch eine Beratungsstelle am Museum entstehen, die die Open Access Zeitschriften am Haus verwaltet, deren Nutzung fördert und eine allgemeine Beratungsfunktion zum Thema Open Access publishing haben soll.



\subsection{Open-Science Initiativen anstoßen}
%Welche Möglichkeiten siehst du, um (eigene) Open Science-Initiativen ​ an deiner wissenschaftlichen Einrichtung anzustoßen? ​Welche Herausforderungen und Chancen bestehen dort, um das Thema Open Science sichtbar/noch sichtbarer zu machen?
Das Museum ist mit den verschiedenen Forschungsbereichen zu Informationswissenschaft und auch Wissenschaftskommunikation prinzipiell bestens aufgestellt, um sich mit dem Thema Open Science intensiver zu befassen. Es gibt jedoch ein paar Punkte, die dem Thema noch im Weg stehen:
1. Auch wenn Open Science von jedem Wissenschaftler selbst mitgelebt werden muss, sollte das Institut eine gewisse Strategie vorgeben. Als Teil der Leibniz-Gemeinschaft sollte das Vorgehen hierzu zumindest innerhalb der Leibniz-Institute abgestimmt sein, Leibniz beschäftigt sich (zumindest nach außen) lediglich mit Open Access. Die Zuständigkeiten für den Anstoß könnten also hin- und hergeschoben werden bzw. zu einem Abwarten führen.
2. Personalmittel sind am Haus sehr knapp und kaum unbefristet, die Auslastung der Angestellten in den in Frage kommenden Forschungsbereichen ist bereits sehr hoch. Neue Initiativen sind daher schwierig zu integrieren und müssten vermutlich über Drittmittel zuerst beantragt werden.
3.  Der Aspekt Citizen Science ist sehr präsent am Haus und wird auch stark gefördert. Jedoch ist es in erster Linie die Forschung am Thema Citizen Science selbst, aber keine Duchführung von tatsächlichen Projekten, mit denen Forschungsdaten gewonnen werden können. Die Beteiligung der Naturwissenschaftler ist entsprechend gering in diesem Bereich, auch sind viele Projekte einfach nicht für Citizen Science geeignet. Durch starke Präsenz von Citizen Science besteht meiner Ansicht nach die Möglichkeit, dass Open Science in der Wahrnehmung der Naturwissenschaftler auf dieses eine Thema reduziert wird.


\subsection{Interesse an Open-Science im Umfeld} % Fachbereich, Umfeld?
In meiner Arbeitsgruppe gab es von Anfang an Interesse an der Thematik. In Gesprächen mit Mitgliedern der verschiedenen Fachbereiche war das Interesse an dem Fellow-Programm groß.

\subsection{Open Science in deiner Forschung}
%Inwiefern war das Fellow-Programm nützlich für dich, um aktiv(er) ​ Open Science in deiner Forschung ​ anzuwenden? Welche persönlichen Erfolge konntest du erzielen oder auch nicht? Wie möchtest du daran langfristig anknüpfen? ​
Das Programm hat mir einen massiven Zugewinn an Wissen zur Außenkommunikation von Projekten gebracht. Dazu gehören auch die unangenehmeren Teile wie Lizenzfragen und technische Umsetzbarkeit. Diese Punkte fallen definitiv leichter unter Anleitung und mit einem Ansprechpartner. Ein persönlicher Erfolg ist für mich die Projektwebsite, da ich hiervon langfristig profitieren werde und diesen Aspekt unbedingt fortsetzen möchte. Die externe Motivation und Anleitung, neue Dinge auszuprobieren war für mich auch wichtig, da gerade der Forschungsalltag schon sehr arbeitsintensiv ist und viele Ansätze leider oft als reine gute Intentionen enden und nicht umgesetzt werden. 
Mein langfristiges Ziel ist es, Open Science mit inhaltlicher, naturwissenschaftlicher Arbeit fest zu vereinbaren. Die zeitliche Ressourcenverteilung müsste für mich klar strukturiert sein, da ich während des Programms häufig das Gefühl hatte, eine umfassende Beschäftigung mit dem Thema würde meine ganze Arbeitszeit einnehmen, da das Thema so umfangreich ist. Ohne inhaltlichen Fortschritt gibt es jedoch nichts mehr zu kommunizieren / zu öffnen, daher halte ich für Naturwissenschaftler eine Beschränkung auf die für sich persönlich wichtigsten Aspekte für unumgänglich.


\subsection{Ansprechperson Open Science}
% Im Rahmen der Programmevaluation ​ planen wir, uns nach Ablauf der Programmlaufzeit nach dem aktuellen Stand von Open Science an den Institutionen der Fellows zu erkundigen. Wer ist hierfür an deiner Institution die entsprechende Ansprechperson?
Es gibt hierfür zwei Ansprechstellen am Museum; Dr. Katrin Volland als Leiterin des Forschungsbereichs Wissenskommunikation und Wissensforschung: katrin.vohland@mfn-berlin.de, sowie Dr. Jana Hoffmann als kommissarische Leiterin des Forschungsbereichs Digitale Welt und Informationswissenschaft: jana.hoffmann@mfn-berlin.de .




\end{document}
